\newcommand{\myvar}[1]{{{\mathit{#1}}}}
\newcommand{\block}{\myvar{block}}
\newcommand{\chunks}{\myvar{chunk}}
\newcommand{\proofs}{\myvar{proof}}
\newcommand{\ctext}{\myvar{ctext}}
\newcommand{\chash}{\myvar{chash}}
\newcommand{\bhash}{\myvar{bhash}}
\newcommand{\comm}{\myvar{comm}}
\newcommand{\coinp}{\ensuremath{\mathit{coin}'}\xspace}
\newcommand{\timeout}{\myvar{timeout}}
\newcommand{\mendorsement}{\myvar{endorsement}}
\newcommand{\data}{\myvar{data}}
\newcommand{\contract}{\myvar{contract}}
\newcommand{\contractp}{\myvar{contract}'}
\newcommand{\escrow}{\myvar{escrow}}
\newcommand{\database}{\myvar{DB}}
\newcommand{\mproof}{\myvar{MProof}}
\newcommand{\merkle}{\myvar{MHash}}
\newcommand{\merkProve}{\myvar{MProve}}
\newcommand{\merkproof}{\myvar{MProof}}
\newcommand{\merkVer}{\myvar{MVerify}}
\newcommand{\chunk}{\myvar{chunk}}
\newcommand{\ignore}[1]{}


% keys
\newcommand{\skAlice}{\ensuremath{sk_{\mathcal{A}}}}
\newcommand{\pkAlice}{\ensuremath{pk_{\mathcal{A}}}}
\newcommand{\pkBob}{\ensuremath{pk_{\mathcal{B}}}}
\newcommand{\skBob}{\ensuremath{sk_{\mathcal{B}}}}
\newcommand{\pkTTP}{\ensuremath{pk_{Arbiter}}}

\subsection{Buy}
Let Alice be the \emph{buyer} and Bob be the \emph{seller}.
As a precondition, the
protocol assumes that Alice already knows $\bhash$, which is the root of the Merkle hash tree computed on the block $\block$ that Alice wants to buy.

\subsubsection{Merkle tree}
First, we give the procedure $\merkle(h,\block,\ell)$ to compute
$\bhash$, the Merkle hash function that outputs the root of the
depth-$\ell$ Merkle tree of a given $\block$ with hash function $h$.
Next, we give (1) the procedure to compute the proof
$\merkProve((h,\block,\ell),i)$ that a particular value 
is the $i^\mathit{th}$ leaf of the Merkle hash tree of which
$\bhash$ is the root; the procedure 
$\merkVer((h,\ell),(\bhash,i,\chunk))$ to verify the proof $\mproof$ that
 the value
$\chunk$ is the $i^\mathit{th}$ leaf of the Merkle hash tree of which
$\bhash$ is the root.


First, let us explain the idea of the $\merkle$ procedure.  Consider a rooted
binary tree with $2^\ell$ leaves (\emph{i.e.} it is a binary tree of
height $\ell$).  Associated with every node the tree, there is the
address $a$ of the node.  Specifically, the label $a$ associated with
the root node is the empty string $\varepsilon$. The label of a left
(resp. right) child is derived by concatenating $0$ (resp., $1$) to
the label of the parent node. Thus, the label $a$ associated with the
$i^{\mathit{th}}$ leaf is the integer $i$ written in binary.  Stored
at a node labeled with $a$, is a value $v_a$.  In a Merkle tree,
stored at each leaf $0\leq a\leq 2^\ell-1$ is the value $v_a =
h(chunk_a)$, where $\chunk_a$ is the $a^\mathit{th}$ chunk of the block; 
and stored at every internal node $0\leq a \leq 2^i-1$ at
depth $i\geq 1$ is the value $v_a=h(v_{a||0}||v_{a||1})$.  Finally,
the value corresponding to the root node is $v_\varepsilon =
h(v_0||v_1)$.  The algorithm $\merkle$ outputs this value.

\begin{algorithm}[H]\label{alg:merkle}
\SetKwComment{Comment}{}{}
\SetKwInput{Pre}{Pre-conditions}
\SetKwInput{Post}{Post-conditions}
\dontprintsemicolon

\KwIn{A collision-resistant hash function $h~:~\{0,1\}^*\mapsto \{0,1\}^\hashLength$, the data block $\block$, the desired tree height $\ell$.}
\Pre{None.}
\KwOut{The value $\bhash$.}
\Post{None.}
\BlankLine



\Indp
  If $\ell=0$, output $h(\block)$. \;
  Otherwise, divide the block $\block$ into $\block_0$ and $\block_1$, $2$ strings of approximately equal byte lengths (this has to be done in a deterministic fashion so that running twice on same input gives same results) and output $h(\merkle(h,\block_0,\ell-1)||\merkle(h,\block_1,\ell-1))$\;
\Indm

\caption{$\merkle$: Generating a Merkle hash tree.}
\end{algorithm}


To prove that $\chunk$ is the $i^\mathit{th}$ chunk of the block
represented by $\bhash=\merkle(\block)$, reveal the values $v_{a_j}$
where for $1 \leq j \leq \ell$, the label $a_j$ is obtained by taking
the first $j-1$ bits of the binary representation of $i$, and
concatenating to them the negation of the $j^\mathit{th}$ bit. (For
example, if $i = 0101$, then $a_1 = 1$, $a_2=00$, $a_3=011$ and $a_4 =
0100$.)  In the Merkle tree, the node $a_j$ will be the sibling of the
$j^\mathit{th}$ node on the path from the root to the chunk in
question.  The procedure for doing this is as follows:


\begin{algorithm}[H]\label{alg:merkleproof}
\SetKwComment{Comment}{}{}
\SetKwInput{Pre}{Pre-conditions}
\SetKwInput{Post}{Post-conditions}
\dontprintsemicolon

\KwIn{A collision-resistant hash function $h~:~\{0,1\}^*\mapsto \{0,1\}^\hashLength$, the data block $\block$, the desired tree height $\ell$; the index $i$.}
\Pre{$0\leq i < 2^\ell$}
\KwOut{The value $\merkproof$.}
\Post{None.}
\BlankLine


\Indp
  If $\ell=0$, return $\block$. \;
  Otherwise, divide the block $\block$ into $\block_0$ and $\block_1$, $2$ strings of approximately equal byte lengths (this has to be done in a deterministic fashion, the same as what is done by the algorithm $\merkle$). \;
  If $i\geq 2^{\ell-1}$, return $(\merkle(h,\block_0,\ell-1),\merkProve((h,\block_1,\ell-1),i-2^{\ell-1}))$.\;
  Else return $(\merkle(h,\block_1,\ell-1),\merkProve((h,\block_0,\ell-1),i))$.\;
\Indm

\caption{Generating the proof $\merkproof=(v_1,\ldots,v_j,\chunk)$ that $\chunk$ is the $i^\mathit{th}$ leaf of the Merkle tree.}
\end{algorithm}

To verify that $(v_1,...,v_\ell)$ is a valid proof that $\chunk$ is the $i^\mathit{th}$ chunk of $\block$
associated with $\bhash$, 
we recompute the labels $a_j$ (as above).  We know that $v_j = v_{a_j}$ is the value that should be associated with node labelled
$a_j$.  Let $i=i_1i_2\ldots i_\ell$, \emph{i.e.} $i_j$ is the $j^\mathit{th}$
bit of the $\ell$-bit binary representation of $i$.
Let $b_j = i_1\ldots i_j$ be the $j$-bit prefix of $i$.
First, we know that $v_i = h(\chunks)$.
For each $j$, $\ell-1\geq j \geq 0$, compute $v_{b_j}=h(v_{b_j||0}||v_{b_j||1})$.
We can do it because one of $(v_{b_j||0},v_{b_j||1})$ is $v_{a_{j+1}}$,
and the other one is computed in the previous step.
Finally, verify that $v_\varepsilon = \bhash$.  The pseudocode for this procedure is as follows:

\begin{algorithm}[H]\label{alg:merkleverify}
\SetKwComment{Comment}{}{}
\SetKwInput{Pre}{Pre-conditions}
\SetKwInput{Post}{Post-conditions}
\dontprintsemicolon

\KwIn{A collision-resistant hash function $h~:~\{0,1\}^*\mapsto \{0,1\}^\hashLength$, 
the value $\bhash$, the desired tree height $\ell$; the proof $\merkproof$, the index $i$.}
\Pre{$0\leq i < 2^\ell$}
\KwOut{Accept or reject.}
\Post{None.}
\BlankLine

TO BE WRITTEN

\caption{Verifying the proof $\merkproof=(v_1,\ldots,v_j,\chunk)$ that $\chunk$ is the $i^\mathit{th}$ leaf of the Merkle tree.}
\end{algorithm}


If two conflicting proofs can be constructed (\emph{i.e.} for
$\chunks\neq\chunks'$, there are proofs that
each of them is the $i^\mathit{th}$ chunk of $\block$
associated with $\bhash$), then a collision in $h$
is found, contradicting the assumption that $h$ is collision-resistant. 

\subsubsection{The Buy protocol}
We present our protocol that lets the Buyer buy a file block from the Seller.
Before the start of the protocol, the Buyer has (1) $\bhash = \merkle(\block)$;
and (2) has withdrawn a wallet from the Bank and is ready to create unendorsed coins together with their endorsements.
from a trusted authority (\emph{i.e.} tracker).  Alice and Bob will agree on a
$\timeout$ by when Bob must provide Alice with the block.  The protocol
works as follows:

\ignore{
\begin{enumerate}

\item
Bob chooses a symmetric key $K$ and sends $\ctext =\break \Encrypt{K}{\block}$ to Alice.

\item 
  Alice constructs an endorsed e-coin $(\coinp,\mendorsement)$.
  Alice calculates $\chash = \merkle(h,\ctext,\ell)$,
  chooses a random value $r$ and calculates the exchange ID
    $v = h(r)$.
  Alice sets $\contract = (\bhash,\chash,\timeout,\coinp,v)$.
  She escrows the $\mendorsement$ under the arbiter's public-key:
  $\escrow = \TVEncrypt{Arbiter}{\mendorsement}{\contract}$.
  Alice sends Bob $(\coinp, \contract, \escrow)$.

\item
Bob verifies that $(\coinp, \contract, \escrow)$ is formed correctly. If he is s
atisfied, he establishes a secure connection to Alice using standard techniques 
and sends the key $K$.  Otherwise, Bob terminates.

%\item[*]$\quad B \rightarrow A : \quad K$

\item %4.
If Alice receives a $K$ that lets her decrypt
  $\ctext$ correctly before $\timeout$, she responds with
  $\mendorsement$. Otherwise, Alice waits until $\timeout$ and then
  calls $\AliceResolve(r)$ on the arbiter, as in 
  Algorithm~\ref{sec:protocol}.\ref{alg_AliceResolve}.

%\item[*]$\quad A \rightarrow B : \quad \mendorsement$

\item %5.
If Bob does not receive a correct $\mendorsement$ 
%that lets him deposit $\coinp$
  before $\timeout$, he calls $\BobResolve(K, \escrow, \contract)$
  on the arbiter, as 
  in Algorithm~\ref{sec:protocol}.\ref{alg_BobResolve}.
\end{compactenum}

\subsubsection{The buyer's resolution algorithm}
\subsubsection{The seller's resolution algorithm}
\subsubsection{The arbiter's algorithm}
}
