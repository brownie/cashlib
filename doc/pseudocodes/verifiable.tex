\section{Verifiable Encryption}
We will be using Caminisch-Shoup verifiable encryption \cite{csVE}. This will later be the verifiable escrow method used in fair exchange.

\textsc{Assumptions}:\\
The security of the scheme relies on Strong RSA assumption, Paillier's Decision Composite Residuosity assumption \cite{paillierENC}, and existence of a collision-resistant family of hash functions.


\textsc{Parameters}:\\
The parameter $m$ denotes the number of messages that can be verifiably encrypted at once. For Endorsed E-Cash, we need $m=3$.



\textsc{Setup}:\\
The key generation of CS verifiable encryption uses two runs of special RSA group generation as in Algorithm \ref{setupRSA}. The important differences are clearly identified below:
\begin{itemize}
\item Let $N$ be the modulus returned by the first special RSA group generation. The generators will be picked in a completely different way. In fact, the group that will be in use will be a subgroup of $Z_{N^2}^{*}$ instead of $Z_{N}^{*}$.
\item Pick a random number $f'$ from $Z_{N^2}^{*}$. Compute $f = f'^{2n} \mod N^2$.
\item Pick random numbers $x_1,\ldots,x_m,y,z$ from the interval $(0,N^2 / 4)$. Compute $a_i = f^{x_i} \mod N^2$ for $1 \leq i \leq m$, $d = f^{y} \mod N^2$, $e = f^{z} \mod N^2$.
\item Compute $b = (1+N) \mod N^2$. Notice that $b$ is an element of order $N$ in $Z_{N^2}^{*}$.
\\
\item Let the second special RSA group generation return a group with modulus $n$ and generators $g_1,\ldots,g_m,h$.
\\
\item Also, pick a key $\hk$ as the key for a keyed hash function. For simplicity in presentation below, we will omit the key in hashes, but it will be used for every hash calculation.
\\
\item The public key (\ie encryption and verification key) is composed of all public parts of the groups, and the hash function key: $VEPK = N,a_1,\ldots,a_m,b,d,e,f,n,g_1,\ldots,g_m,h,\hk$.
\item The secret key (\ie decryption key) is the secret parts of the groups $VESK = P,Q,x_1,\ldots,x_m,y,z,p,q$ such that $N=PQ$ and $n=pq$.
\end{itemize}


The absolute value algorithm below will be used in the verifiable encryption scheme a lot.


\begin{algorithm}[H]\label{absVE}
\SetKwComment{Comment}{}{}
\SetKwInput{Pre}{Pre-conditions}
\dontprintsemicolon

\KwIn{$x$ in range $(0,N^2)$}
\Pre{$x$ needs to be in range $(0,N^2)$}
\KwOut{$\abs(x)$}
\BlankLine

\Comment{abs} \;
\Indp
  \If{$x > N^2 / 2$}
    {Output $(N^2 - x) \mod N^2$ \;}
  \Else
    {Output $x \mod N^2$ \;}
\Indm

\caption{Absolute Value procedure for Camenisch-Shoup verifiable encryption scheme.}
\end{algorithm}





\subsection{Encrypt}
The encryption procedure creates a ciphertext. To turn it into verifiable encryption, we will need some specialized non-interactive zero knowledge proofs. Just as we need for fair exchange, this is a labeled encryption scheme \cite{LABELED-ENC}. The \textit{label} is also known as \textit{tag}, or \textit{contract}.



\begin{algorithm}[H]\label{encryptVE}
\SetKwComment{Comment}{}{}
\SetKwInput{Pre}{Pre-conditions}
\SetKwInput{Post}{Post-conditions}
\dontprintsemicolon

\KwIn{Verifiable encryption public key $VEPK = N,a_1,\ldots,a_m,b,d,e,f,n,g_1,\ldots,g_m,h,\hk$, messages to be verifiably encrypted $x_1,\ldots,x_m$, a label $L$}
%\Pre{}
\KwOut{Ciphertext $u_1,\ldots,u_m,v,w$.}
%\Post{}
\BlankLine

\Comment{Encrypt} \;
\Indp
  Pick a random number $r$ from the interval $(0,N / 4)$. \;
  \For{$i$ : $1..m$}
    {Compute $u_i = b^{x_i} * a_{i}^{r} \mod N^2$. \;}
  Compute $v = f^r \mod N^2$ \;
  Compute $w = \abs([d * e^{\hash(u_1 \cc \ldots \cc u_m \cc v \cc L)}]^r \mod N^2)$ \;
  Output $u_1,\ldots,u_m,v,w$. \;
\Indm

\caption{Encryption procedure for Camenisch-Shoup verifiable encryption scheme. This procedure is run by the Encryptor. This is a sub-procedure; it's not verifiable yet.}
\end{algorithm}




\subsection{Verifiably Encrypt}
We will now provide a conversion from regular encryption to verifiable encryption for Camenisch-Shoup verifiable encryption scheme. This will involve non-interactive zero knowledge proofs.

This verification can be used to prove correct encryption of discrete logarithms. When we consider Endorsed E-Cash, the Prover will verifiably encrypt $x_1,x_2,r_y$ such that $y = g_1^{x_1} * g_2^{x_2} * f^{r_y}$ in the group chosen by the Endorsed E-Cash setup. Take a note that it is a different group than the ones used in verifiable encryption here. $X$ below will be $y$ of the Endorsed E-Cash.


\begin{algorithm}[H]\label{verifiablyEncryptVE}
\SetKwComment{Comment}{}{}
\SetKwInput{Pre}{Pre-conditions}
\SetKwInput{Post}{Post-conditions}
\dontprintsemicolon

\KwIn{Verifiable encryption public key $VEPK = N,a_1,\ldots,a_m,b,d,e,f,n,g_1,\ldots,g_m,h,\hk$, commitment $X$ to all messages $x_1,\ldots,x_m$ (or a commitment $X_i$ to each one of them), a label $L$}
%\Pre{}
\KwOut{Ciphertext and its proof $u_1,\ldots,u_m,v,w,X',\pi_{VE}$.}
%\Post{}
\BlankLine

\Comment{Verifiably Encrypt} \;
\Indp
  Get the ciphertext $u_1,\ldots,u_m,v,w$ from the Encrypt function in Algorithm \ref{encryptVE}, along with the randomness $r$ used in the encryption. \;
  Pick a random number $s$ from the interval $(0,N / 4)$. \;
  Compute a Fujisaki-Okamoto commitment $X'$ to $x_1,\ldots,x_m$ using the randomness $s$ in the group defined by $n,g_1,\ldots,g_m,h$ (the second RSA group generated for verifiable encryption purposes). \;
  Prove knowledge of $r,s,x_1,\ldots,x_m$ such that $v^2 = f^{2r} \mod N^2$, $w^2 = [d * e^{\hash(u_1 \cc \ldots \cc u_m \cc v \cc L)}]^{2r} \mod N^2$, and that each $u_{i}^2 = b^{2 x_i} * a_{i}^{2r} \mod N^2$, and that each $x_i$ corresponds to the one in $X'$ and $X$, and that the range of each $x_i$ is $(-N/2,N/2)$ (SpecialRange). Call this proof $\pi_{VE}$. \;
  Output $u_1,\ldots,u_m,v,w,X',\pi_{VE}$. \;
\Indm

\caption{Verifiable Encryption procedure for Camenisch-Shoup verifiable encryption scheme. This procedure is run by the Encryptor/Prover.}
\end{algorithm}



Any party with the verifiable encryption public key can verify the encryption.



\begin{algorithm}[H]\label{verifyVE}
\SetKwComment{Comment}{}{}
\SetKwInput{Pre}{Pre-conditions}
\SetKwInput{Post}{Post-conditions}
\dontprintsemicolon

\KwIn{Verifiable encryption public key $VEPK = N,a_1,\ldots,a_m,b,d,e,f,n,g_1,\ldots,g_m,h,\hk$, commitment $X$ (or commitments $X_1,\ldots,X_m$), a label $L$, ciphertext and its proof $u_1,\ldots,u_m,v,w,X',\pi_{VE}$}
%\Pre{}
\KwOut{\accept or \reject.}
%\Post{}
\BlankLine

\Comment{Verify} \;
\Indp
  Verify $\pi_{VE}$. If verification fails, output \reject. \;
  \If{$\abs(w) = w \mod N^2$}
    {Output \accept. \;}
  \Else
    {Output \reject. \;}
\Indm

\caption{Verification procedure for Camenisch-Shoup verifiable encryption scheme. This procedure is run by the Verifier.}
\end{algorithm}





\subsection{Decrypt}
Only a party equipped with the correct secret key can decrypt. In the fair exchange, this party will be the Arbiter.


\begin{algorithm}[H]\label{decryptVE}
\SetKwComment{Comment}{}{}
\SetKwInput{Pre}{Pre-conditions}
\SetKwInput{Post}{Post-conditions}
\dontprintsemicolon

\KwIn{Verifiable encryption public key $VEPK = N,a_1,\ldots,a_m,b,d,e,f,n,g_1,\ldots,g_m,h,\hk$, associated secret key $VESK = P,Q,x_1,\ldots,x_m,y,z,p,q$, a label $L$, ciphertext $u_1,\ldots,u_m,v,w$}
\KwOut{Plaintext $m_1,\ldots,m_m$ or \error.}
%\Post{}
\BlankLine

\Comment{Decrypt} \;
\Indp
  \If{$\abs(w) \neq w \mod N^2$ OR $v^{y + z * \hash(u_1 \cc \ldots \cc u_m \cc v \cc L)} \neq w^2 \mod N^2$}
    {Output \error. \;}
  Compute $t = 2^{-1} \mod N$. \;
  \For{$i$ : $1..m$}
    {Compute $m'_i = (u_i / v^{x_i})^{2t} \mod N^2$. \;
     Set $m_i = (m'_i - 1) / N \mod N^2$. \;
     \If{$m_i$ is not in range $(0,N)$}
       {Output \error. \;}}
  Output $m_1,\ldots,m_m$ \;
\Indm

\caption{Decryption procedure for Camenisch-Shoup verifiable encryption scheme. This procedure is run by the Decryptor/Arbiter.}
\end{algorithm}




\subsection{Prove Decryption}
This part is complicated, and for now, will not be employed since the Arbiter in the fair exchange will be a trusted entity.
