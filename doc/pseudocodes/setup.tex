\section{Setup}
In our system, we will use three different types of groups: special RSA
groups, prime-order groups, and Paillier groups (i.e., groups of the form
$\Z_{N^2}^*$).  In this section we focus on the former two group types, as the
Paillier groups are used only in a superficial way for verifiable encryption
and are not considered parameters of the system.  Our groups are usually 
generated by a trusted third party, but 
we also describe methods that can be used if for some reason the setup is 
untrusted.

In what follows, let $QR_n$ denote the group of quadratic residues modulo $n$.

\begin{algorithm}[H]\label{setupRSA}
\SetKwComment{Comment}{}{}
\SetKwInput{Pre}{Pre-conditions}
\SetKwInput{Post}{Post-conditions}
\dontprintsemicolon

\KwIn{security parameter $\RSALength$, number of generators $m$}
\Pre{$\RSALength$ must be at least $1024$. $m$ must be at least $1$.}
\KwOut{modulus $n$, primes $p,q,p',q'$, generators $g_1,\ldots,g_m,h$, exponents $a_1,\ldots,a_m$}
\Post{The length of $n$ must be $\RSALength$. $n=pq$, $p=2p'+1$, $q=2q'+1$. $p,q,p',q'$ are primes. $p = 3 \mod 4$, $q = 3 \mod 4$. $\vert p \vert = \vert q \vert = \RSALength/2$}
\BlankLine

\Comment{Special RSA Group Setup} \;
\Indp
  Choose primes $p,q$ of length $\RSALength/2$ each such that $p = 2p'+1$ and
$q = 2q'+1$ where $p',q'$ are primes (so $p,q$ are safe primes, $p',q'$ are
Sophie Germain primes, and $n$ is a special RSA modulus). Note that we will
need $p = 3 \mod 4$ and $q = 3 \mod 4$ in order to have $n$ be a Blum
integer, and that this will indeed be the case for primes chosen as above. \;
  Compute $n = pq$. \;
  Choose $h \leftarrow QR_n$.  To do this, pick a random residue, square it, and check that $h^{(p-1)/2} = 1 \mod p$ and $h^{(q-1)/2} = 1 \mod q$. Alternatively, pick $h_p \leftarrow Z_{p}^*$ and $h_q \leftarrow Z_{q}^*$, and set $h = h_{p}^{q-1} * h_{q}^{p-1} \mod n$. \;
  \For{$i$ : $1..m$}
    {Choose $a_i \leftarrow \{0,1\}^{\RSALength + \stat}$ and set $g_i = h^{a_i} \mod n$ \;}
  Output modulus $n$, primes $p,q,p',q'$, generators $g_1,\ldots,g_m,h$, and exponents $a_1,\ldots,a_m$. \;
\Indm

\caption{Setup for generating a special RSA group.}
\end{algorithm}


\begin{algorithm}[H]\label{setupPrime}
\SetKwComment{Comment}{}{}
\SetKwInput{Pre}{Pre-conditions}
\SetKwInput{Post}{Post-conditions}
\dontprintsemicolon

\KwIn{security parameters $\primeLength, \orderLength$, number of generators $m$}
\Pre{$\primeLength$ must be at least $512$. $\orderLength$ must be at least
$160$ or $2*\stat$, whichever is larger.\footnote{The best algorithms to solve
Discrete Logarithm problem in prime-order groups have running time that
depends on the size of either the order of the subgroup or the modulus of the
group \cite{dlAlgorithms,mov:handbook:97}} $m$ must be at least $1$.}
\KwOut{modulus $\primeMod$, order $\primeOrder$, generators $g_1,\ldots,g_m,h$, exponents $a_1,\ldots,a_m$}
\Post{The length of $\primeMod$ must be $\primeLength$. The length of $\primeOrder$ must be $\orderLength$.}
\BlankLine

\Comment{Prime Order Group Setup} \;
\Indp
  Pick a prime order $\primeOrder$ of length $\orderLength$ and a prime modulus $\primeMod$ of length $\primeLength$ where $\primeMod-1$ is divisible by $\primeOrder$. This results in the order $\primeOrder$ subgroup of $Z_{\primeMod}^*$. \;
  Pick generator $h$ for the subgroup with order $\primeOrder$.  To do this, pick $h' \leftarrow Z_{\primeMod}^*$ and set $h =  h'^{(\primeMod - 1) / \primeOrder} \mod \primeMod $. \;
  \For{$i$ : $1..m$}
    {Pick generators $g_i$ each with order $\primeOrder$ (same method as above)\;}
  Output modulus $\primeMod$, order $\primeOrder$, generators $g_1,\ldots,g_m,h$, exponents $a_1,\ldots,a_m$ \;
\Indm

\caption{Setup for generating a prime-order group.}
\end{algorithm}


\textsc{Definitions of Groups}:\\
Throughout this paper, the ``definition of the RSA group'' means the modulus $n$ and possibly the bases $g_1,\ldots,g_m,h$ if they are not explicitly defined.

Similarly, the ``definition of the prime-order group'' means the modulus $\primeMod$ and the order $\primeOrder$, and possibly the bases $g_1,\ldots,g_m,h$ if they are not explicitly defined.

Bases will generally be explicitly defined in the algorithms.


\textsc{Group Operations}:\\
A group operation is an operation on a group element. We will use multiplicative notation when denoting the group operation and inverse of an element. All group operations must be done modulo the modulus of the group. Therefore, all group operations will be done modulo $n$ for the RSA group, and modulo $\primeMod$ for the prime-order group. A simple example of this is $g*h \mod n$.



\textsc{Operations on Exponents}:\\
An operation on exponents means the multiplication, addition, or inversion of exponents. In RSA groups, all such operations must be performed over integers, as the order of the group is not assumed to be known. In prime-order groups, all such operations will be done modulo $\primeOrder$, as this value is always public.  An example of such an operation is $g^{a*b \mod \primeOrder} \mod \primeMod$.


\textsc{Group Secrets}:\\
When we emphasize that some party has not generated a group itself, we mean that that party may not know the secrets associated with the group generation. For an RSA group, the secrets will be $p,q$; the factorization of the modulus $n$. Furthermore, in both an RSA group and a prime-order group, the relative discrete logarithms of the generators are secrets.
