\subsection{Fujisaki-Okamoto Commitment Scheme}

\textsc{Overview}:\\
The Fujisaki-Okamoto commitment scheme~\cite{fo:statistical-zk:c97,df:commitment:ac02} is a statistically hiding, computationally binding commitment scheme. The Committer commits to something and sends the resulting commitment to the Verifier. At some later time, the Committer opens the commitment and the Verifier needs to verify that the opening matches the commitment sent before.


\textsc{Setup}:\\
This commitment scheme uses a special RSA group.  In the case that an untrusted party (\eg the Verifier) generates the RSA group, he needs to prove to the Committer that each $g_i$ is in the group generated by $h$, so that the commitment is statistically hiding. This can be done by proving in zero knowledge the knowledge of $a_i$ such that $g_i = h^{a_i} \mod n$. The committer may not generate or know $p, q, p', q', a_1,\ldots,a_m$, as otherwise the scheme will not provide any meaningful binding property.


\textsc{Assumptions}:\\
The security of the scheme relies on the Strong RSA assumption.


\textsc{Hiding}:\\
The hiding property relies on the fact that $g_i,h$ all generate the same group, so that when randomization is used the resulting commitment is a random group element.


\textsc{Binding}:\\
The binding property relies on the assumption that the Committer cannot find two different openings (using the same bases) that result in the same commitment. This follows from the Strong RSA assumption.


\textsc{Warnings}:\\
This commitment scheme requires at least two generators: $g_1,h$, otherwise it does not work.



\begin{algorithm}[H]\label{commitFO}
\SetKwComment{Comment}{}{}
\SetKwInput{Pre}{Pre-conditions}
\SetKwInput{Post}{Post-conditions}
\dontprintsemicolon

\KwIn{Definition of the RSA group, number of secrets $k$, bases $g_1,\ldots,g_k,h$, secrets $x_1,\ldots,x_k$}
\Pre{The RSA group must be generated by a trusted third party or it must be proven that each $g_i,h$ generates $QR_n$.}
\KwOut{commitment $C$, opening $open$}
%\Post{}
\BlankLine

\Comment{Commit} \;
\Indp
  Pick a random number $r$ from $\{0,1\}^{\RSALength + \stat}$. \;
  Create the commitment $C = h^r \prod_{i=1}^{k} g_i^{x_i} \mod n$. \;
  Output commitment $C$ and opening $open = x_1,\ldots,x_k,r$. \;
\Indm

\caption{Commitment procedure of the Fujisaki-Okamoto commitment scheme. This procedure is run by the Committer.}
\end{algorithm}


\textsc{Comments}:\\
The random number $r$ actually needs to be relatively prime to $\phi(n)$, but since this will be the case with high probability we omit the check.
\\

To open a commitment, the Committer just sends the opening $open$ to the Verifier. After that, the Verifier needs to verify the opening of the commitment. This is done as follows:
\\

\begin{algorithm}[H]\label{verifyFO}
\SetKwComment{Comment}{}{}
\SetKwInput{Pre}{Pre-conditions}
\SetKwInput{Post}{Post-conditions}
\dontprintsemicolon

\KwIn{Definition of the RSA group, number of secrets $k$, bases $g_1,\ldots,g_k,h$, commitment $C$, opening $open = x_1,\ldots,x_k,r$}
\Pre{The RSA group may NOT be generated by the Committer.}
\KwOut{$\accept$ or $\reject$}
%\Post{}
\BlankLine

\Comment{Verify} \;
\Indp
  \If{$C = h^r \prod_{i=1}^{k} g_i^{x_i} \mod n$}
    {Output $\accept$ \;}
  \Else
    {Output $\reject$ \;}
\Indm

\caption{Verification procedure of the Fujisaki-Okamoto commitment scheme. This procedure is run by the Verifier after receiving the opening $open$ for a commitment $C$.}
\end{algorithm}

