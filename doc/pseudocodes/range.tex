\subsection{Proof that a committed value lies within an interval [lo,hi]}

We would like to prove that a secret value $x$ lies within a publicly known interval $[lo,hi]$, as in \cite{boudotInterval,lipmaaInterval}. The protocol uses a special RSA group. The verifier knows a commitment $C_x$ to $x$, and $lo,hi$, along with the RSA group definition. The prover additionally knows the secret $x$.

Observe that $x$ is in the interval $[lo,hi]$ iff $x-lo \geq 0$ and $hi-x \geq 0$. Furthermore, any non-negative integer can be represented as a sum of 4 squares \cite{lipmaaInterval} (\eg $x-lo = v_{1}^{2} + v_{2}^{2} + v_{3}^{2} + v_{4}^{2}$), whereas negative integers cannot. These numbers ($v_i$) can be computed efficiently \cite{lipmaaInterval,rabinShallit}. Let $W_i$ be $v_{i}^{2}$.

Further observe that a commitment to $x-lo$ can easily be obtained by computing $C_x g_{1}^{-lo}$, and a commitment to $hi-x$ can easily be obtained by computing $g_{1}^{hi} C_{x}^{-1}$.

We have seen that using the Mult algorithm, we can prove that a committed number is actually a square. Therefore, the basic idea is to prove that each $W_i$ is a square (by committing to them so that the verifier does not learn $W_i$ or $v_i$), and then their sum is equal to $x-lo$. A similar proof also needs to be given for $hi-x$.

We call this protocol \textbf{Range}. It uses Fujisaki-Okamoto commitments. Therefore it requires the same setup and the same assumptions. Everything said for the Fujisaki-Okamoto commitments also applies here.


\textsc{Assumptions}:\\
The security of the scheme relies on Strong RSA assumption.


\textsc{Honest-Verifier Zero Knowledge}:\\
The protocol is honest verifier zero knowledge provided that the Fujisaki-Okamoto commitment is hiding. This means that the RSA group must be generated by a trusted third party or it must be proven to the Prover that each $g_i,h$ generates $QR_n$.


\textsc{Soundness}:\\
The extraction works under the Strong RSA assumption, which requires that the RSA group may NOT be generated by the Prover.
\\


Below we provide the pseudocode for proving that a given value $y$ is non-negative ($\geq 0$). We call this protocol \textbf{Non-Negative}. Then, proving the interval means running this protocol twice, one for $x-lo$ and one for $hi-x$.

In the Non-Negative protocol, the Prover is given a commitment $C_y$ and its opening $open_y = y,r_y$, and the group definition. The Verifier is given $C_y$ and the group definition.

\begin{enumerate}
 \item The Prover computes $v_1,v_2,v_3,v_4$ using the four-squares algorithm \cite{lipmaaInterval,rabinShallit}. Let $W_i = v_{i}^{2}$.
 \item The Prover computes commitments $C_i$ to $W_i$ with their openings $open_i = W_i,r_i$, with the only rule that the last randomness is set to be $r_4 = r_y - (r_1+r_2+r_3)$.
 \item The Prover proves that each $C_i$ is a commitment to a square, using the special case of the Mult protocol.
 \item The Verifier, in addition to checking that each $C_i$ is a commitment to a square, checks if $C_y = \Pi_{i=1}^{4} C_i$. If all checks pass, the Verifier accepts, otherwise he rejects.
\end{enumerate}

